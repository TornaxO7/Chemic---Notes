\documentclass[../../main.tex]{subfiles}
\begin{document}
\subsection{Superabsorber}
%
% 1)
\subsubsection{Was versteht man unter Polymerisation?}
Bei einer \textit{Polymerisation} werden Moleküle mit einer Doppelbindung zu
Makromolekülen, indem sie ihre Doppelbindung \say{aufgeben} und mit weiteren
Molekülen zu größeren Molekülen werden.
%
% 2)
\subsubsection{}
Noch nicht gemacht
%
% 3)
\subsubsection{Zeichne einen sinnvollen Strukturformelausschnitt des Superabsorbers mit bindenden und nichtbindenden Elektronenpaaren.}
\begin{align*}
    \schemestart
        \chemfig{
            \charge{180=\.}{C}
                ( -[2]H)
                ( -[6]H)
            -[0]C
                ( -[2]H)
                ( -[6]O
                % first carboxyl
                    ( =[7]\charge{22.5=\|, -68.5=\|}{O})
                    ( -[5]\charge{135=\|, -45=\|}{O}
                        ( -[::0]H)
                    )
                )
            -[0]C
                ( =[2]\charge{45=\|, 135=\|}{O})
            -[0]C
                ( -[2]H)
                ( -[6]H)
            -[0]C
                ( -[2]H)
                % second carboxyl
                ( -[6]C
                    ( =[7]\charge{22.5=\|, -68.5=\|}{O})
                    -[5]\charge{135=\|, 225=\|, 315=\|, 270:2.5mm=$\ominus$, 250:7mm=\ce{Na^{$\oplus$}}}{O}
                )
            -[0]C
                ( -[2]H)
                ( -[6]H)
            % sub-connection, direction: Down
            -[0]C
                ( -[2]H)
                % first C atom downstairs
                ( -[6]C
                    ( =[7]\charge{22.5=\|, -68.5=\|}{O})
                    ( -[5]X
                        -[6]C
    		                ( -[8]H)
    		                ( -[4]H)
                        -[6]C
    		                ( -[8]H)
    		                ( -[4]H)
                        -[6]C
    		                ( -[8]H)
    		                ( -[4]H)
                        -[6]C
    		                ( -[8]H)
    		                ( -[4]H)
                        -[6]C
    		                ( -[8]H)
    		                ( -[4]H)
                        -[6]X
                        -[7]C
                            ( =[2]\charge{135=\|, 45=\|}{O})
                        -C
                            % second connection downstairs
                            ( -[2]H)
                            ( -[6]C
                                ( -[0]H)
                                ( -[4]H)
                                -[6]\charge{0:1mm=\.}{C}
                            		( -[4]H)
                                -[6]C
                            		( =[7]\charge{275=\|, 365=\|}{O})
                            		( -[5]\charge{337=\|, 517=\|}{O}
                                		( -[6]H)
                            		)
                            )
                        % Connection to the right
                        -[1]C
                    		( -[2]H)
                    		( -[-2]H)
                        -[0]C
                    		( -[2]H)
                    		( -[-2]H)
                        -[0]\charge{0:1mm=\.}{C}
                            ( -[6]H)
                        -[2]C
                    		( =[3]\charge{95=\|, 185=\|}{O})
                    		( -[1]\charge{157=\|, 337=\|}{O}
                        		( -[2]H)
                    		)
                    )
                )
            -[0]C
    		    ( -[2]H)
    		    ( -[-2]H)
            -[0]\charge{0:1mm=\.}{C}
                ( -[2]H)
                ( -[6]C
    		        ( =[7]\charge{275=\|, 365=\|}{O})
    		        ( -[5]\charge{337=\|, 517=\|}{O}
        	        	( -[6]H)
    		        )
                )
        }
    \schemestop
    \\
\end{align*}
\begin{center}
    $X$ ist irgendein Molekül, wie zum Beispiel \ce{O} oder \ce{NH}.
\end{center}
% 4)
\subsubsection{Was kannst du über den Vernetzungsgrad des Superabsorbers aussagen? In welche Kunststoffklasse kann man ihn einordnen?}
Noch nicht gemacht
% 5)
\subsubsection{Erläutere mit Fachsprache, wie der neutrale pH-Wert zustande kommt. Wie nennt man ein solches System?}
Die Acrylsäure besitzt eine Carboxylgruppe, die ein Proton abgeben kann und
somit den pH-Wert senken lässt. Das Natriumcrylat besitzt ein Natrium, welches
in der Lösung ein Proton aufnimmt, sodass der ph-Wert steigt. Da beide in die
entgegengesetzte Richtung des pH-Wertes gehen, neutralisieren sie sich. So ein
System nennt man \textit{Puffer-System}.
%
%
% END
\end{document}
