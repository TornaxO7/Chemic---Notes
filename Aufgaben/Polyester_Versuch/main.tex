\documentclass[../../main.tex]{subfiles}
\begin{document}
\begin{center}
    19.10.20
\end{center}
\subsubsection{Polyester --- Schülerversuch}
\begin{align*}
    \schemestart
        2n
        \chemname{
            \chemfig{
                H
                -C
                    ( -[2]H)
                    (-[6]\charge{0=\|, 180=\|}{O}
                        (-[6]H)
                    )
                -C
                    ( -[2]H)
                    (-[6]\charge{0=\|, 180=\|}{O}
                        -[6]H
                    )
                -C
                    ( -[2]H)
                    (-[6]\charge{0=\|, 180=\|}{O}
                        -[6]H
                    )
                -H
            }
        }
        {Glycerin}
        \+
        3n
        \chemname{
            \chemfig{
                H
                -\charge{240=\|, 60=\|}{O}
                -[7]C
                    (=[5]\charge{180=\|, 270=\|}{O})
                -[0]C
                	( -[2]H)
                	( -[-2]H)
                -[0]C
                	( -[2]H)
                	( -[-2]H)
                -[0]C
                	( =[1]\charge{90=\|, 0=\|}{O})
                	( -[-1]\charge{60=\|, -120=\|}{O}
                		 ( -[0]H)
                	)
            }
        }
        {Bernsteinsäure (Butandisäure)}
    \schemestop
\end{align*}
\begin{align*}
    \schemestart
    \arrow{->[Kondensationsreaktion]}[,2]
    \chemname{
        \chemfig{
            \Charge{90=$\vdots$}{\vphantom{C}}
            -[-2]\charge{0=\|, 180=\|}{O}
            -[6]C
            	( -[8]H)
            	( -[4]H)
            -[6]C
            	( -[8]H)
            	( -[4]H)
            -[6]C
                (=[0]\charge{45=\|, -45=\|}{O})
            -[-2]\charge{0=\|, 180=\|}{O}
            -[-2]C
                (-[4]C
                	( -[6]H)
                	( -[2]H)
                    -[4]\charge{90=\|, -90=\|}{O}
                    -[4]\Charge{180=$\ldots$}{\vphantom{C}}
                )
            -[0]C
            	( -[2]H)
            	( -[-2]H)
            -\charge{90=\|, -90=\|}{O}
            -C
                (=[2]\charge{135=\|, 45=\|}{O})
            -[0]C
            	( -[2]H)
            	( -[-2]H)
            -[0]C
            	( -[2]H)
            	( -[-2]H)
            -C
                (=[6]\charge{315=\|, 225=\|}{O})
            -\Charge{0=$\ldots$}{\vphantom{C}}
        }
    }
    {Ein Duroplast}
    \schemestop
\end{align*}
%------------------
\underline{c1}
\begin{align*}
    \schemestart
        n
        \arrow{,0}[0,0]
        \chemfig{
            H
            -[-1]\charge{45=\|, -135=\|}{O}
            -[-1]C
                (=[1]\charge{135=\|, -45=\|}{O})
            -[6]C
            	( -[0]\charge{90=\|, -90=\|}{O}
                    (-H)
                )
            	( -[4]H)
            -[6]C
            	(-[8]H)
            	(-[6]H)
            	(-[4]H)
        }
        \arrow{->}
        \arrow{,0}[0,0]
        \chemname{
            \chemfig{
                -[@l]C
                    (=[2]\charge{135=\|, 45=\|}{O})
                -[0]C
                	( -[2]H)
                	( -[-2]{\ce{CH3}})
                -\charge{90=\|, -90=\|}{O}
                -[@r]
            }
            \polymerdelim[delimiters={[]}, height=45pt, depth=45pt, indice=n]{l}{r} 
        }
        {Polymilchsäure}
        \+
        n \ce{H2O}
    \schemestop
\end{align*}
%------------------
\underline{c2}
\begin{align*}
    \schemestart
        \chemfig{
            C
            	(=[5]\charge{270=\|, 180=\|}{O})
            	(-[3]\charge{240=\|, 60=\|}{O}
            		 ( -[4]H)
            	)
            -C
            	( -[2]{OH})
            	( -[-2]H)
            -C
            	( -[2]H)
            	( -[-2]H)
            -C
            	(=[1]\charge{90=\|, 0=\|}{O})
            	(-[-1]\charge{60=\|, -120=\|}{O}
            		 ( -[0]H)
            	)
        }
    \schemestop
    \text{und}
    \hspace{1cm}
    \schemestart
        \chemfig{
            {H2C}
                (-[-2]{OH})
            -{CH}
                (-[-2]{OH})
            -{CH2}
                (-[-2]{OH})
        }
    \schemestop
    \\
    \\
    \schemestart
        \text{ergibt}
        \hspace{1cm}
        \chemfig{
            \ldots
            -[@l]C
                (=[2]\charge{135=\|, 45=\|}{O})
            -C
            	( -[2]H)
            	( -[-2]\charge{0=\|, 180=\|}{O}
                    (-[-2]\vdots)
                )
            -[0]C
            	( -[2]H)
            	( -[-2]H)
            -C
                (=[2]\charge{135=\|, 45=\|}{O})
            -\charge{90=\|, -90=\|}{O}
            -[0]C
            	( -[2]H)
            	( -[-2]H)
            -C
            	( -[2]H)
            	( -[-2]\charge{0=\|, 180=\|}{O}
                    (-[-2]\vdots)
                )
            -C
                (=[2]\charge{135=\|, 45=\|}{O})
            -[@r]\Charge{0=$\ldots$}{\vphantom{C}}
        }
        \polymerdelim[delimiters={[]}, height=45pt, depth=45pt, indice=n]{l}{r} 
    \schemestop
\end{align*}
Es ist dort kein Reaktionspfeil, weil es zu viele kombinationsmöglichkeiten
gibt und wir deswegen nur ein einzelnes Beispiel genommen haben.

\underline{c3}
\begin{align*}
    \schemestart
        \chemname{
            \chemfig{
                {H2C}
                    (-[-2]{COOH})
                -[,2]{CH}
                    (-[-2]{COOH})
                -[,2]{CH}
                    (-[2]{OH})
                    (-[-2]{COOH})
            }
        }
        {Zitronensäure}
    \schemestop
\end{align*}
Da wir 3 Carboxylgruppen und nur \textit{eine} Hydroxygruppe, können nur kurze
Ketten entstehen da nicht alle Carboxylgruppen eine Verbindung eingehen können
(weil wir nur eine Hydroxygruppe haben).
%
%
% END
\end{document}
