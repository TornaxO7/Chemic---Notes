\documentclass[../../main.tex]{subfiles}
\begin{document}
\begin{center}
    19.10.20
\end{center}
\subsubsection{Polyester --- Schülerversuch}
\begin{align*}
    \schemestart
        2n
        \chemname{
            \chemfig{
                H
                -C
                    ( -[2]H)
                    (-[6]\charge{0=\|, 180=\|}{O}
                        (-[6]H)
                    )
                -C
                    ( -[2]H)
                    (-[6]\charge{0=\|, 180=\|}{O}
                        -[6]H
                    )
                -C
                    ( -[2]H)
                    (-[6]\charge{0=\|, 180=\|}{O}
                        -[6]H
                    )
                -H
            }
        }
        {Glycerin}
        \+
        3n
        \chemname{
            \chemfig{
                H
                -\charge{240=\|, 60=\|}{O}
                -[7]C
                    (=[5]\charge{180=\|, 270=\|}{O})
                -[0]C
                	( -[2]H)
                	( -[-2]H)
                -[0]C
                	( -[2]H)
                	( -[-2]H)
                -[0]C
                	( =[1]\charge{90=\|, 0=\|}{O})
                	( -[-1]\charge{60=\|, -120=\|}{O}
                		 ( -[0]H)
                	)
            }
        }
        {Bernsteinsäure (Butandisäure)}
    \schemestop
\end{align*}
\begin{align*}
    \schemestart
    \arrow{->[Kondensationsreaktion]}[,2]
    \chemname{
        \chemfig{
            \Charge{90=$\vdots$}{\vphantom{C}}
            -[-2]\charge{0=\|, 180=\|}{O}
            -[6]C
            	( -[8]H)
            	( -[4]H)
            -[6]C
            	( -[8]H)
            	( -[4]H)
            -[6]C
                (=[0]\charge{45=\|, -45=\|}{O})
            -[-2]\charge{0=\|, 180=\|}{O}
            -[-2]C
                (-[4]C
                	( -[6]H)
                	( -[2]H)
                    -[4]\charge{90=\|, -90=\|}{O}
                    -[4]\Charge{180=$\ldots$}{\vphantom{C}}
                )
            -[0]C
            	( -[2]H)
            	( -[-2]H)
            -\charge{90=\|, -90=\|}{O}
            -C
                (=[2]\charge{135=\|, 45=\|}{O})
            -[0]C
            	( -[2]H)
            	( -[-2]H)
            -[0]C
            	( -[2]H)
            	( -[-2]H)
            -C
                (=[6]\charge{315=\|, 225=\|}{O})
            -\Charge{0=$\ldots$}{\vphantom{C}}
        }
    }
    {Ein Duroplast}
    \schemestop
\end{align*}
%
%
% END
\end{document}
