\documentclass[a4paper, 12pt]{scrartcl}

% -----------
% Packages
% -----------
\usepackage[ngerman]{babel}
\usepackage[utf8]{inputenc}
\usepackage[T1]{fontenc}				% west european chars
\usepackage{amsmath, amsthm, amssymb} 	% math symbols
\usepackage{geometry} 					% Ermöglicht die Formatierung des Seitenlayouts
%\usepackage{showframe}
%\usepackage{siunitx}	% simpler variables
\usepackage{comment}  	% To comment out things
\usepackage{textcomp} 	% For arrow
\usepackage[export]{adjustbox}[2011/08/13] 	% very big images
\usepackage{subcaption} % for subfigures
\usepackage[version=4]{mhchem} % Chemic stuff

\usepackage{tocloft} 	% show dots in the table of contents
\usepackage{chemfig} % Chemic structures
\usepackage{tabularx}			  % Better tables 
\usepackage[column=0]{cellspace}  % Adding horizontal/vertical space to cells
\usepackage{makecell}			  % for creating cells with better customicing
\usepackage{adjustbox}
\usepackage{xcolor}
\usepackage{tcolorbox}
\usepackage{gensymb}

\usepackage{enumitem} % able to change the settings of enumerates/itemizes

% -----------
% commands
% -----------
\renewcommand{\cftsecleader}{\cftdotfill{\cftdotsep}}

\setlength\cellspacetoplimit{8pt}
\setlength\cellspacebottomlimit{8pt}

\renewcommand{\tabularxcolumn}[1]{m{#1}}

\def\RED{\gdef\printatom##1{\color{red}\ensuremath{\mathrm{##1}}}}
\def\BLUE{\gdef\printatom##1{\color{blue}\ensuremath{\mathrm{##1}}}}
\def\BLACK{\gdef\printatom##1{\color{black}\ensuremath{\mathrm{##1}}}}

% ------------------
%  own environments
% ------------------
\newtcolorbox{merke}[1]{
    colback = red!5!white,
    colframe = red!75!black,
    fonttitle = \bfseries,
    title = #1
}

\newtcolorbox{Notiz}[1]{
    colback = blue!5!white,
    colframe = blue!75!black,
    fonttitle = \bfseries,
    title = #1
}

% -----------
% Settings 
% -----------
% border on the side
\geometry {
        left = 2.5cm,
        right = 2.5cm,
        top = 1.0cm,
        bottom = 0.5cm,
        includeheadfoot
}

\title{Chemie --- Zusammenfassung}
\author{TornaxO7}
\date{\today}

\parskip5pt
\parindent0pt

\begin{document}
\maketitle
\tableofcontents
%------------------
\section{Flex --- Begriffe}
\begin{itemize}
    \item Viskosität == Zähflüssig
\end{itemize}
%------------------
\section{}
Allgemeine Struktur:
\begin{align*}
    \schemestart
        \chemname{
            \chemfig{
                C
                	( =[3]\charge{180=\|, 90=\|}{O})
                	( -[1]\charge{150=\|, -30=\|}{O}
                		 ( -[2]H)
                	)
                -[6]C
                    ( -[4]\charge{180=\|}{N}
                        ( -[2]H)
                        ( -[6]H)
                    )
                    ( -[0]H )
                -[6]R
            }
        }
        {L --- $\alpha$ --- Aminosäure}
    \schemestop
\end{align*}
%------------------
\subsection{Aminosäuren}
Das $\alpha$ steht für die \textit{Carboxylgruppe am benachbartem C---Atom}.

Aminosäuren liegen als \textbf{Zwitter} vor.
\begin{itemize}
    \item Durch \textbf{Carboxylgruppe}: Kann \textbf{Sauer (Protonendonator)}
        reagieren.
    \item Durch \textbf{Aminogruppe}: Kann \textbf{Basisch (Protonenakzeptor)}
        reagieren.
\end{itemize}
Es bildet durch die beiden Gruppen eine \textbf{intramolekulare
    Protonenwanderung}.
%------------------

\begin{tabularx}{\linewidth}{
	|>{\centering\arraybackslash\hspace{0pt}}X
	|>{\centering\arraybackslash\hspace{0pt}}X
	|>{\centering\arraybackslash\hspace{0pt}}X
	|}
	\hline
    Kation
    &
    Zwitterion
    &
    Anion
    \\ % ----------
    \hline
    \schemestart
        \chemfig{
            COOH
            -[6]C
                (-[4]H_{3}N^{+})
                (-[0]H)
            -[6]H
        }
    \schemestop
    &
    \schemestart
        \chemfig{
            COO^{-}
            -[6]C
                (-[4]H_{3}N^{+})
                (-[0]H)
            -[6]H
        }
    \schemestop
    &
    \schemestart
        \chemfig{
            COO^{-}
            -[6]C
                (-[4]H_{2}N)
                (-[0]H)
            -[6]H
        }
    \schemestop
    \\ % ----------
    \hline
\end{tabularx}
%------------------

Den pH---Wert, an dem die Aminosäuren hauptsächlich als Zwitterion vorliegen
nennt man \textbf{isoelektrischen Punk (IEP)}.
%
\subsection{Proteine}
\subsubsection{Peptidbindung und Polypeptide}
Bei einer Peptidbindung spalten sich ein Sauerstoff von der Carboxylgruppe und
zwei Wasserstoff Atome von der Aminogruppe ab, sodass Wasser entsteht.
Anschließend verbinden sie sich:
\begin{align*}
    \schemestart
        \chemname{
            \chemfig{
                H
                -\charge{45:5pt=$\oplus$}{N}
                    ( -[2]H)
                    ( -[6]H)
                -[0]C
                	( -[2]H)
                	( -[-2]H)
                -[0]C
                	( =[1]\charge{90=\|, 0=\|}{O})
                -[7]\charge{90=\|, 0=\|, -90=\|, 45:5pt=$\ominus$}{O}
            }
        }
        {Glycin}
        \+
        \chemname{
            \chemfig{
                H
                -N
                    ( -[2]H)
                    ( -[6]H)
                -C
                	( -[2]H)
                    ( -[-2]{CH_{3}})
                -C
                	( =[1]\charge{90=\|, 0=\|}{O})
                -[7]\charge{90=\|, 0=\|, -90=\|, 45:5pt=$\ominus$}{O}
            }
        }
        {Alanin}
    \schemestop
\end{align*}
\begin{align*}
    \schemestart
        \arrow{->}
        \chemname{
            \chemfig{
                \BLUE H_{3}N^{+}\BLACK
                -C
                    ( -[2]H)
                    ( -[-2]CH_{3})
                -C
                    ( =[2]\charge{135=\|, 45=\|}{O})
                -\charge{90=\|}{N}
                    ( -[-2]H)
                -C
                    ( -[2]H)
                    ( -[-2]CH_{3})
                -\RED COO^{-}\BLACK
            }
        }
        {Gly --- Ala}
    \schemestop
\end{align*}
\begin{itemize}
    \item \textcolor{blue}{N --- Terminales Ende}
    \item \textcolor{red}{C --- Terminales Ende}
\end{itemize}
Reaktionstyp heißt: \textbf{Kondensationsreaktion} (Wasser wird abgespalten.)\\
\textbf{Polypeptide} sind zusammenverbundene \textbf{Peptidbindungen}.
%------------------
\subsubsection{Sekundär-, Tertiär- und Quartärstruktur}
\begin{itemize}
    \item \textbf{Primärstruktur}

        Primärstruktur = Reihenfolge der einzelnen (durch Peptidbindung
        verknüpften) Aminosäuren, die das Protein aufbauen.

    \item \textbf{Sekundärstruktur}

        \begin{itemize}
            \item beschreibt regelmäßig räumliche wiederholende \textbf{Strukturelemente}
            \item Regelmäßigkeit entsteht durch Wasserstoffbrücken der 
                \chemfig{
                    C=O
                }
                und der
                \chemfig{
                    N-H
                }
                Gruppe.
            \item Proteine besitzt viele \textit{Wasserstoffbrücken}\\
                $ \rightarrow$ Starker Zusammenhalt im Molekül
        \end{itemize}

    \item \textbf{Tertiärstruktur}

        Darstellung der \textbf{Wechselwirkungen} zwischen den
        Aminosäureresten.\\
        %
        \textbf{Echte Bindungen:}
        \begin{enumerate}
            \item \textcolor{yellow}{Disulfidbrücken} (entstehen, wenn zwei
                Cysteinreste miteinander reagieren)
            \item \textcolor{blue}{Ionenbindung} zwischen funktionellen Gruppen
        \end{enumerate}
        %
        \textbf{Zwischenmolekulare Kräfte}
        \begin{enumerate}
            \item \textcolor{red}{Wasserstoffbrücken}
            \item \textcolor{gray}{Van --- der --- Waals --- Kräfte}
        \end{enumerate}

    \item \textbf{Quartärstruktur}

        \begin{itemize}
            \item Eine gemeinsame Funktionseinheit aus mehreren
                Proteinmolekülen.
            \item Gleiche Bindungskräfte wie bei der Tertiärstruktur halten
                Proteinketten zusammen.
        \end{itemize}
\end{itemize}
%------------------
\subsubsection{$\alpha$ Helix und $\beta$ Faltblatt}
\begin{itemize}
    \item \textbf{$\alpha$ Helix}

        \begin{itemize}
            \item sehr große Aminosäureresten winden sich schraubenförmig um
                seine Längenachse
            \item Zusammenhalt der Schraubenform durch \textbf{intramolekulare Wasserstoffbrücken}
            \item Windungen sind \textbf{rechtsgängig} (wie beim Korkenzieher)
            \item Aminosäurereste weisen nach außen
        \end{itemize}

    \item \textbf{$\beta$ Faltblatt}

        \begin{itemize}
            \item beruht auf \textbf{intermolekulare Wasserstoffbrücken}
                zwischen Proteinketten
            \item Aminosäurereste abwechselnd unter- und oberhalb der
                Peptidgruppenebene
            \item $\alpha$-Helices und $\beta$-Faltblattstrukturen sind oft
                Nebeneinander im Proteinmolekül.
        \end{itemize}
\end{itemize}
\begin{merke}{\textbf{Intra}molekulare und \textbf{Inter}molekulare
        Wasserstoffbrücken}
    \begin{itemize}
        \item \textbf{Intermolekular:}

            Ein Vorgang (z.B. chemische Reaktion) \textbf{zwischen}
            verschiedenen Molekülen.
        \item \textbf{Intramolekular:}

            Ein Vorgang \textbf{innerhalb} eines einzelnen Moleküls.
    \end{itemize}
\end{merke}
%------------------
\subsubsection{Zusammenhaltende Kräfte}
\begin{itemize}
    \item \textbf{Sekundärstruktur}: Durch \textbf{Disulfidbrücken}
    \item \textbf{Tertiär- und Quartärstrukturen} durch
        \textbf{Wasserstoffbrücken oder Ionen --- Bindung}
\end{itemize}
%------------------
\subsubsection{Denauturierung}
\underline{Allgemeine Definition:}

Veränderung der Umgebungsbedingungen\\
$ \rightarrow$ Umfaltungen\\
$ \rightarrow$ Strukturänderungen\\
$ \rightarrow$ Protein verändert sich und verliert seine Funktion. \textbf{Die
    Strukturänderungen} können reversibel oder irreversibel sein.

\subsubsection{Denauturierungsmechanismen}
\begin{itemize}
    \item \textbf{Erhitzen}

        Wärmebewegung überwinden zwischenmolekulare Kräfte\\
        $ \rightarrow$ Formieren sich neu.

    \item \textbf{Alkohole, Gerbstoffe, inerte Salze}

        Sekundär- und Quartärstrukturen werden aufgrund der Konkurrenz (mit
        Alkoholen, Gerbstoffen, etc.) um Wasserstoffbrücken zerstört.
        \begin{Notiz}{Konservierung}
            Salz oder Alkohol kann zur konservierung von protonhaltigen Esswaren
            verwendet werden. Mit Tannin kann man Leder gerben.
        \end{Notiz}

    \item \textbf{Änderung des ph --- Wertes}

        Verhinderung von salzartigen Bindungen zwischen $\ce{NH3^{+}}$ - und
        $\ce{COO^{-}}$ - Resten

    \item \textbf{Fällung durch Schwermetall --- Ionen (Tertiärstruktur)}

        Vor allem betroffen sind: \textbf{Schwefel --- und Stickstoffhaltige
            funktionelle Gruppen}. Durch mehrwertige Metall --- Ionen entstehen
        \textbf{Quervernetzungen} zwischen verschiedenen Proteinmolekülen und
        damit zur Ausflockung.

    \item \textbf{Fällung durch Tenside}

        Grenzflächende Substanzen stören die apolaren Bindungen im Protein

    \item \textbf{Salze}

        Verlust der Hydrathülle (Anlagerungen von Wassermolekülen)

    \item \textbf{Radioaktive Strahlung}
\end{itemize}

\subsubsection{Proteinnachweis}
\begin{itemize}
    \item \textbf{Farbreaktion (am wichtigsten)}

        \underline{Stichwort:} \textbf{Biuretreaktion}\\
        \underline{Vorgang:} Alkalische Eiweißlösung + Kupfer(||)-sulfat-Lösung
        $ \rightarrow$ violettte Lösung.

    \item \textbf{Xanthoproteinreaktion}

        Eiweiß + Salpetersäure $ \rightarrow$ Gelbfärbung
\end{itemize}

%------------------
\subsection{Enzyme}
\subsubsection{Schlüssel --- Schloss --- Prinzip}
\begin{itemize}
    \item Chemische Reaktionen findem im \textbf{aktivem Zentrum} statt.
    \item Verbindung zwischen Substrat und Enzym: \textbf{Enzym --- Substrat ---
            Komplex}
    \item Enzyme reagieren auf \textbf{ein ganz bestimmtes Substrat}
    \item Reaktionsname zwischen Enzym und Substrat: \textbf{Schlüssel ---
            Schloss --- Prinzip} $ \rightarrow$ Eigenschaftsname:
        \textbf{Substratspezifität}
\end{itemize}

\subsubsection{Beeinflussung der Katalyseaktivität}
\begin{enumerate}
    \item \textbf{Abhängigkeit der Temperatur}

        \textbf{Denauturierung} bei über $40\degree C$.

    \item \textbf{Abhängigkeit vom pH --- Wert}

        \textbf{Tertiärstruktur} hängt von sauren und alkalischen
        Aminosäurebausteinen ab. \\
        Zugabe von $\ce{H3O^{+}_{(aq)}}$ oder $\ce{OH^{-}_{(aq)}}$ \\
        $\rightarrow$ Ionenbindungen werden gestört \\
        $ \rightarrow$ damit auch das \textbf{aktive Zentrum}

    \item \textbf{Abhängigkeit der Konzentration}

        \begin{itemize}
            \item \textbf{Substratsättigung:} Alle Enzyme sind beschäftigt \\
                $\rightarrow$ Erhöhung der Substratkonzentration \\
                $ \rightarrow$ kein Anstieg der Reaktionsgeschwindigkeit
            \item \textbf{Substrathemmung:} Zu viele Substatmoleküle lagern sich
                am aktivem Zentrum \\
                $ \rightarrow$ Verlangsamung
        \end{itemize}
\end{enumerate}
%------------------
\subsection{DNA}
\subsubsection{Allgemein}
\begin{itemize}
    \item Speichert die Erbinfo im Zellkern
    \item Nukleotid: Desoxyribose + \ce{PO4} + Base
    \item Nukleosid: Desoxyribose + Base
    \item Nukleotidsequenz: Ist in der m-RNA; Der genetische Code
    \item Gen: Eine Informationseinheit in einem Abschnitt eines DNA-Moleküls
    \item Genetischer Code:
        \begin{itemize}
            \item Abfolge von drei Nucleotiden codiert eine bestimme Aminosäure
            \item $4^{3} = 64$ Codeworte
            \item Die meisten Aminosäuren besitzen mehrere Codeworte
        \end{itemize}
    \item Basenpaare: A + T, C + G
\end{itemize}
%
Mithilfe des Strickleitermodels:
\begin{itemize}
    \item Holmen (die Seitensträngen): Abwechselnde \textbf{Desoxyribose-} und
        \textbf{Phosphorsäureeinheiten} \\
        $ \rightarrow$ Esterbindungen
    \item Sprossen: Basenpaare (Adenin, Cytosin, Guanin und Thymin) \\
        $ \rightarrow$ Wasserstoffbrücken
\end{itemize}

\subsubsection{Strukturformeln (auswendig können)}
\begin{align*}
    \schemestart
        \chemname{
            \chemfig{
                H
                -\charge{90=\|, -90=\|}{O}
                -P
                    (-[6]\charge{0=\|, 180=\|}{O}
                        -[6]H
                    )
                    (=[2]\charge{135=\|, 45=\|}{O})
                -\charge{90=\|, -90=\|}{O}
                -H
            }
        }
    {Phosphorsäure ($\ce{H3PO4}$)}
    \chemname{
        \chemfig[cram width=4pt]{
            H
            -\charge{90=\|, -90=\|}{O}
            -C
                (-[2]H)
                (-H)
            -[-2]C?
                (-[-2]H)
            <[7,1.5]C
                (-[2]H)
                ( -[-2]\charge{180=\|, 0=\|}{O} 
                    (-[-2]H)
                )
            -[,1.5,,,line width=4pt]C
                (-[2]H)
                (-[-2]\charge{0=\|, 180=\|}{O}
                    (-[-2]H)
                )
            >[1,1.5]C
                (-[-2]H)
                ( -[2]\charge{180=\|, 0=\|}{O}
                    (-[2]H)
                )
            -[:150,2]\charge{45=\|, 135=\|}{O}?
        }
    }
    {$\beta$ --- D --- Ribose}
    \schemestop
\end{align*}
\begin{align*}
    \schemestart
        \chemname{
        \chemfig[cram width=4pt]{
            H
            -\charge{90=\|, -90=\|}{O}
            -C
                (-[2]H)
                (-H)
            -[-2]C?
                (-[-2]H)
            <[7,1.5]C
                (-[2]H)
                ( -[-2]\charge{180=\|, 0=\|}{O} 
                    (-[-2]H)
                )
            -[,1.5,,,line width=4pt]C
                (-[2]H)
                (-[-2]H)
            >[1,1.5]C
                (-[-2]H)
                ( -[2]\charge{180=\|, 0=\|}{O}
                    (-[2]H)
                )
            -[:150,2]\charge{45=\|, 135=\|}{O}?
        }
    }
    {$\beta$ --- D --- 2 --- Desoxyribose}
    \schemestop
\end{align*}
%
\subsubsection{Verknüpfungen}
\begin{itemize}
    \item Phosphorsäure mit $\beta$ --- D --- 2 --- Desoxyribose\\
        $ \rightarrow$ \textbf{Veresterung}
    \item $\beta$ --- D --- 2 --- Desoxyribose mit einer Base\\
        $ \rightarrow$ \textbf{Kondensationsreaktion}
    \item zwischen zwei Basen\\
        $ \rightarrow$ \textbf{Wasserstoffbrücken}
\end{itemize}
%------------------
\subsection{Aromaten}
\subsubsection{Benzol}
Eigenschaften:
\begin{itemize}
    \item Wasserklar
    \item leicht beweglich
    \item stark lichtbrechend
    \item Siedetemperatur: $80{,}1\degree C$
    \item Fest bei $5{,}5\degree C$
    \item Dichte: $0{,}875 \frac{g}{cm^{3}}$
    \item Hydrophob
    \item Hydrophobil (liebt Hydrophobe Stoffe)
    \item brennt mit leuchtend, stark rußender Flamme
\end{itemize}
%
Gesundheitsproblematik: 
\begin{itemize}
    \item Ist giftig und Krebserregend
    \item schädigt Leber und Knochenmark
    \item Kann Leukämie auslösen
\end{itemize}
%
Vorkommen:
\begin{itemize}
    \item Nebenprodukt beim Verkoken von Steinkohle
\end{itemize}
%
Verwendung:
\begin{itemize}
    \item Aufheizen von Kokskammern
    \item In Steinkohleteer
\end{itemize}
%
Kekule und Entdeckung des Benzol:
\begin{itemize}
    \item Damals in Glaslaternen, Rest: Öliges Kondensat
    \item Faraday erkennt 1:1 Gemisch zwischen C und H
    \item Keine Isomere von Benzol\\
        $ \rightarrow$ Alle Kohlenstoffatome sind gleichwertig\\
        $ \rightarrow$ Kekules Strukturformel setzt sich durch.
\end{itemize}
%
Reaktion von Benzol mit Brom im Verglelich zur Reaktion mit Alkenen:

\textbf{Substitution:}
\begin{align*}
    \schemestart
        \chemfig{
            \ldots
            -C
                (-[2]H)
            	(-[-2]H)
            -C
                (-[2]H)
            	(-[-2]H)
            -\Charge{180=$\, \ldots$}{\vphantom{C}}
        }
        \+
        \ce{Br2}
        \arrow{->[Licht]}
        \chemfig{
            \ldots
            -C
                ( -[2]{Br})
            	( -[-2]H)
            -C
            	( -[2]H)
            	( -[-2]H)
            -\Charge{180=$\, \ldots$}{\vphantom{C}}
        }
        \+
        \chemfig{
            H
            -\charge{90=\|, 0=\|, -90=\|}{\ce{Br}}
        }
    \schemestop
\end{align*}
%
\textbf{Addition}
\begin{align*}
    \schemestart
        \chemfig{
            R
            -[7]C
                (-[5]R)
            =C
                (-[1]R)
            -[7]R
        }
        \arrow{0}[,0]
        \+
        \ce{Br2}
        \arrow{->[Licht]}
        \chemfig{
            R
            -C
                (-[2]\charge{180=\|, 90=\|, 0=\|}{Br})
                (-[-2]R)
            -C
                (-[2]\charge{180=\|, 90=\|, 0=\|}{Br})
                (-[-2]R)
            -R
        }
    \schemestop
\end{align*}
%
\subsubsection{Mesomerie}
Allgemein:
\begin{itemize}
    \item Elektronen sind delokalisiert
    \item Delokalisierung ist ein konstanter Dauerzustand
    \item Ist planar, regelmäßiges Sechseck
    \item Bindungsverhältnisse:
        \begin{itemize}
            \item Einfachbindungen zwischen \chemfig{
                H - C
            }
            \item 6 delokalisierte Elektronen
        \end{itemize}
\end{itemize}
%
\textbf{Mesomerieenergie:} \\
Der Energiebetrag, wo ein reales Benzolmolekül stabiler ist, als eine
Grenzformel.

\textbf{Mesomeriestabilität}:\\
Teilchen mit delokalisierten Elektronen sind mesomeriestabilisiert.

Wichtige weitere Aromaten unter dem AB: \textbf{Weitere Aromaten: Bedeutung und
    Benennung}
% ---------------===================---------------
%  Polymerisation
% ---------------===================---------------
\section{Polymerisation}
%
\subsection{7.3.2 Polykondensation}
Kondensationsreaktion:
\begin{center}
    Verknüpfung zweier Moleküle durch Abspaltung eines weiteren Moleküls (z.B.
    Wasser)
\end{center}
%
Bekannte Kondensationsreaktionen:
\begin{enumerate}[label=\alph*)]
    \item Esterbildung (Säure + Alkohol)
    \item Peptidbildung (aus Aminosäuren)
\end{enumerate}
%
Struktureformeln zu a) und b):
%
\begin{enumerate}[label=\alph*)]
    \item \underline{Polyester}

        Möglichkeit 1: Hydroxycarbonsäure\\
        z.B.:
        \schemestart
            n
            \chemfig{
                H
                -\charge{90=\|, -90=\|}{O}
                -C
                	( -[2]H)
                	( -[-2]H)
                -[0]C
                	( =[1]\charge{90=\|, 0=\|}{O})
                	( -[-1]\charge{60=\|, -120=\|}{O}
                		 ( -[0]H)
                	)
            }
        \arrow{->}
        \ldots
        \chemfig{
            -[@l]
            -\charge{90=\|, -90=\|}{O}
            -[0]C
            	( -[2]H)
            	( -[-2]H)
            -C
                ( =[2]\charge{135=\|, 45=\|}{O}))
            -
            -[@r]
        }
        \polymerdelim[delimiters={[]}, height=45pt, depth=40pt, indice=n]{l}{r}
        \schemestop
\end{enumerate}
\end{document}
