\documentclass[../../main.tex]{subfiles}
\begin{document}
\begin{enumerate}[label=\alph*)]
    \item \underline{Polyester}

        Möglichkeit 1: Hydroxycarbonsäure\\
        z.B.:
        \\
        \schemestart
            n
            \chemfig{
                H
                -\charge{90=\|, -90=\|}{O}
                -C
                	(-[2]H)
                	(-[-2]H)
                -[0]C
                	(=[1]\charge{90=\|, 0=\|}{O})
                	(-[-1]\charge{60=\|, -120=\|}{O}
                		 (-[0]H)
                	)
            }
        \arrow{->}
        \ldots
        \chemfig{
            -[@l]\charge{90=\|, -90=\|}{O}
            -[0]C
            	(-[2]H)
            	(-[-2]H)
            -C
                (=[2]\charge{135=\|, 45=\|}{O})
            -[@r]\Charge{180=$\ldots$}{\vphantom{C}}
        }
        \polymerdelim[delimiters={[]}, height=45pt, depth=40pt, indice=n]{l}{r}
        \schemestop
        -[2]\Charge{0=$\ldots$}{\vphantom{C}}
\end{enumerate}
\end{document}
