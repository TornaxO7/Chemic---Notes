\documentclass[../main.tex]{subfiles}
\begin{document}
\subsection{Polymerisation}
\underline{Versuch:} Herstellung von Polysterol\\
Skizze:
\begin{figure}[h]
    \centering
    \incfig{skizze-polymerisation}
    \caption{Skizze Polymerisation}
    \label{fig:skizze-polymerisation}
\end{figure}
\\
\underline{Beobachtung:}
\begin{itemize}
    \item Sidet beim erhitzen (auch wenn die Flamme weggenommen wird)
    \item Viskosität nimmt zu
    \item Schäumt beim siden
    \item aufsteigende Dämpfe, Kondensierun im Steigrohr
\end{itemize}
\underline{Definition:} Polymerisation \\
Verknüpfen kleiner Molekülen mit Doppelbindung zu einem Makromoleküle unter
Verlust der Doppelbindung.\\
%

\underline{Gesamtreaktion:}
\begin{align*}
    \schemestart
        \chemfig{
            **6(
                -
                -
                -
                -
                    (-[2]C
                        (-[3]H)
                        (=[0]C
                            (-[1]H)
                            (-[7]H)
                        )
                    )
                -
                -
            )
        }
        \arrow{->}
        \chemfig{
            -[@{BL}]
            -[0]C
                ( -[6]**6(------))
                ( -[2]H)
            -[0]C
           		( -[2]H)
           		( -[-2]H)
                -[@{BR}]
        }
        \polymerdelim[delimiters={[]}, height=45pt, depth=120pt, indice=n]{BL}{BR}
    \schemestop
\end{align*}
\textbf{\underline{Reaktionsmechanismus:}}
\\
\underline{1. Bildung von Radikalen:}
\begin{align*}
    \schemestart
        \chemname{
            \chemfig{[:90]
                **6(
                    -
                        (-[0]C
                        (=[2]\charge{45=\|,135=\|}{O})
                            (-[0]\charge{90=\|,270=\|}{O}
                                (-[0]\charge{90=\|,270=\|}{O}
                                    (-[0]C
                                        (=[2]\charge{45=\|,135=\|}{O})
                                        (-**6(------))
                                    )
                                )
                            )
                        )
                    -
                    -
                    -
                    -
                    -
                )
            }
        }
        {Dibenzoylperoxid}
    \schemestop
    \\
    \schemestart[][west]
        \arrow{->} 2 \hspace{1mm}
    \schemestop
    \schemestart[][west]
        \chemfig{[:90]
            **6(
                -
                    (-[0]C
                        (=[2]\charge{45=\|, 135=\|}{O})
                        (-[0]\charge{0=\., 90=\|, 270=\|}{O})
                    )
                -
                -
                -
                -
                -
            )
        }
    \schemestop
\end{align*}
Es spaltet sich auf, weil die Peroxidgruppe sehr instabil ist.
%
\\
\underline{2. Startreaktion}
\setchemfig{scheme debug=false}
\begin{align*}  
    \schemestart[][west]
    % first molecule
    \chemfig{[:30]
            **6(
                -
                -
                    (-[0]C
                        (=[2]\charge{45=\|, 135=\|}{O})
                        (-[0]\charge{0=\., 90=\|, 270=\|}{O})
                    )
                -
                -
                -
                -
            )
        }
        \arrow{0}[,0]
        \+{0.5cm, 0.5cm}
        \arrow{0}[,0]
        % second molecule
        \chemfig{
            H
            -[7]@{dest1}C
                (-[6]**6(------))
            =[@{double} 0]@{dest2}C
                ( -[1]H)
                ( -[7]H)
        }
    \schemestop
    % electron arrow
    \chemmove {
        \draw[shorten <= 5pt, shorten >= 5pt] (double) .. controls +(270:10mm) and +(180:10mm) .. (dest1);
        \draw[shorten <= 5pt, shorten >= 5pt] (double) .. controls +(270:10mm) and +(0:10mm) .. (dest2);
    }
    \\
    \schemestart
        \arrow{->}
    \schemestop
    % product
    \underbrace{
        \schemestart
            \chemfig{[:90]
                **6(
                    -
                        (-[0]C
                            (-[0]\charge{90=\|, 270=\|}{O}
                                (-[0]C
                                    (-[0]C)
                                    (-[2]H)
                                    (-[6]**6(------))
                                )
                            )
                            (=[2]\charge{45=\|, 135=\|}{O})
                        )
                    -
                    -
                    -
                    -
                    -
                )
            }
        \schemestop
    }_{R}
\end{align*}  
%
\underline{3. Kettenreaktion/Kettenwachstum:}
\begin{align*}
    \schemestart
        % educt
        R \+{0.5cm, 0.5cm}
        \arrow{0}[,0]
        % educt
        \chemfig{
            H 
            -[7]C
                ( -[6]**6(------))
            =[0]C
                ( -[1]H)
                ( -[7]H)
        }
        \arrow
        % product
        \chemfig{
            R
            -[0]C
                ( -[2]H)
                ( -[6]**6(------))
            -[0]\charge{0=\.}{C}
                ( -[2]H)
                ( -[6]H)
        }
    \schemestop
\end{align*}
%
\underline{4. Kettenabbruch:} \\
Verschiedene Möglickeiten, z.B. Rekombination:
\begin{align*}
    \schemestart
        % educt 1
        \chemfig{
            R_{1}
            -C
                ( -[2]H)
                ( -[6]**6(------))
            -\charge{0=\.}{C}
                ( -[2]H)
                ( -[6]H)
        }
        % ----
        \arrow{0}[,0]
        \+{1cm, 1cm}
        \arrow{0}[,0]
        % ----
        % educt 2
        \chemfig{
            \charge{180=\.}{C}
                ( -[2]H)
                ( -[6]H)
            -C
                ( -[2]H)
                ( -[6]**6(------))
            -R_{2}
        }
        % ----
        \arrow
        % ----
        \chemfig{
            R_{1}
            -C
                ( -[2]H)
                ( -[6]**6(------))
            -C
                ( -[2]H)
                ( -[6]H)
            -C
                ( -[2]H)
                ( -[6]H)
            -C
                ( -[2]H)
                ( -[6]**6(------))
            -R_{2}
        }
    \schemestop
\end{align*}
%
Dibenzoylperoxid ist hier Starter, beziehungsweise Radikalbildner und die Zugabe
von vielen Startern führt zu kürzeren Kettenlängen, da viele Ketten gestartet
werden. (Die Kette von der Gesamtreaktion)
%

\underline{Bemerkung / Beispiele zu Polymerisation}
\begin{enumerate}[label=\alph*)]
    %
    % a)
    \item Bekannte Polymerisation \\
        \noindent
        \begin{tabularx}{\linewidth}{|0l|0c|0c|>{\raggedright\arraybackslash\hspace{0pt}}X|}
            Name & Monomer & Polymermolekül & Einsatzbeispiel\\
            \hline
            \makecell[cl]{Polyethen\\ (PE)} 
                    & 
                    \adjustbox{valign=c}{\chemfig{
                        H
                        -[1]C 
                            (-[3]H)
                        =C 
                            ( -[1]H) 
                        -[7]H
                    }} &
                    \schemestart
                            \chemfig{
                                -[@{upleft,0.5},1]C
                                    ( -[2]H)
                                    ( -[-2]H)
                                -C
                                    ( -[2]H)
                                    ( -[-2]H)
                                -[@{upright,0.5},1]
                            }
                            \polymerdelim[delimiters ={[]}, height = 45pt, depth = 40pt, indice = n]{upleft}{upright}
                    \schemestop
                    & Plastiktüten
            \\
            \makecell[cl]{Polypropen\\ (PP)} 
                 & \adjustbox{valign=c}{\chemfig{
                     H
                     -C
                        ( -[2]H)
                        ( -[6]H)
                    -C
                        ( -[2]H)
                    =C
                        ( -[1]H)
                        ( -[7]H)
                 }} & 
                 \schemestart
                        \chemfig{
                            -[@{upleft,0.5},1]C
                                ( -[2]H)
                                ( -[-2]H)
                            -C
                                ( -[2]{CH_3})
                           -[@{right,0.5},1]
                        }
                        \polymerdelim[delimiters ={[]}, height = 45pt, depth = 40pt, indice = n]{upleft}{upright}
                 \schemestop
                 & Flaschendeckel, Brotdosen
            \\
            \makecell[cl]{Polyvinyl-\\chlorid\\ (PVC)}
                & 
                \adjustbox{valign=c}{\chemfig{
                    H
                    -[1]C
                        ( -[3]H)
                    =C
                        ( -[1]Cl)
                    -[7]H
                }} & 
                \schemestart
                        \chemfig{
                            -[@{upleft,0.5},1]C
                                ( -[2]H)
                                ( -[-2]H)
                            -C
                                ( -[6]H)
                                ( -[2]{Cl})
                            -[@{right,0.5},1]
                        }
                        \polymerdelim[delimiters ={[]}, height = 45pt, depth = 40pt, indice = n]{upleft}{upright}
                \schemestop
                & Rohrleitungen, Vinylböden, Schallplatten
            \\ 
            \makecell[cl]{Polytetra-\\fluorethen\\ (PTFE)} 
                & 
                \adjustbox{valign=c}{\chemfig{
                    F
                    -[1]C
                        ( -[3]F)
                    =C
                        ( -[1]F)
                    -[7]F
                }}&
                \schemestart
                        \chemfig{
                            -[@{upleft,0.5},1]C
                                ( -[2]F)
                                ( -[6]F)
                            -C
                                ( -[2]F)
                                ( -[6]F)
                            -[@{right,0.5},1]
                        }
                        \polymerdelim[delimiters ={[]}, height = 45pt, depth = 40pt, indice = n]{upleft}{upright}
                \schemestop
                & Pfannenbeschichtung (Teflon),
                Funktionskleidung (Goretex) 
        \end{tabularx}
    %
    % b)
    \item \underline{Amorph Teilkristallin} \\
        \begin{itemize}
            \item Amorphe Kunststoffe: Glasartig, transparent
            \item Teilkristalline Kunststoffe: Mechanisch Stabiler, nicht klar
                durchsichtig (milchig), wärmebeständig
        \end{itemize}
        % Amorphe Zeichnung
        \begin{figure}[ht]
            \centering
            \incfig{amorph_tteilkristallin_kristallin}
            \caption{amorph-teilkristallin-kristallin-Eigenschaften-Pfeile}
            \label{fig:amorph tteilkristallin kristallin}
        \end{figure}
    %
    % c)
    \item \underline{Weichmacher} \\
        Kleine Moleküle die sich zwischen die Ketten einlagern können $
        \rightarrow$ Mehr Abstand zwischen den Ketten $ \rightarrow$ Geringere
        zwischenmolekulare Kräfte zwischen den Ketten $ \rightarrow$
        Bessere Verschiebbarkeit der Ketten gegeneinander $ \rightarrow$
        Kunststoff wird weicher

        \underline{Problem:}\\
        \begin{itemize}
            \item Weichmachermoleküle können wieder leicht aus den Ketten
                rausgehen: \\
                %
                Weichmachermoleküle können schädlich sein für Mensch
                und Umwelt
            \item Weichmacher wird spröder, weil der Weichmacher raus ist
        \end{itemize}
    %
    % d)
    \item \underline{Monomere mit konjugierten Doppelbindungen} \\
        Bespiel:
        \begin{align*}
            \schemestart
                \chemname{
                    \chemfig{
                        CH3
                        -[1]{CH}
                            ( -[1]H)
                        =[7]C
                            ( -[5]H)
                            -[1]{CH3}
                    }
                }
                {1,3 --- Butdien}
            \schemestop
        \end{align*}
        %
        \begin{align*}
            \schemestart
                n
                \arrow{0}[,0]
                \chemfig{
                    H
                    -[7]@{l1}C
                        ( -[5]H)
                    =[@{main1}0]@{r1}C
                        ( -[1]H)
                    -[7]@{l2}C
                        ( -[5]H)
                    =[@{main2}0]@{r2}C
                        ( -[1]H)
                    -[7]H
                }
                \arrow{->}
                \chemfig{
                    -[@{BL}0]C
                		( -[2]H)
                		( -[-2]H)
                    -[0]C
                        ( -[2]H)
                    =[0]C
                        ( -[2]H)
                    -[0]C
                   		( -[2]H)
                   		( -[-2]H)
                    -[@{BR}0]
                }
                \polymerdelim[delimiters={[]}, height=45pt, depth=40pt,
                indice=n]{BL}{BR}
            \schemestop
            \chemmove {
                \draw[shorten <= 5pt, shorten >= 5pt] (main1) .. controls
                +(90:1cm) and +(180:1cm) .. (l1);
                \draw[shorten <= 5pt, shorten >= 5pt] (main1) .. controls
                +(90:1cm) and +(0:1cm) .. (r1);
                \draw[shorten <= 5pt, shorten >= 5pt] (main2) .. controls
                +(90:1cm) and +(180:1cm) .. (l2);
                \draw[shorten <= 5pt, shorten >= 5pt] (main2) .. controls
                +(90:1cm) and +(0:1cm) .. (r2);
            }
        \end{align*}
        % But 1,3 ---- dien
        Man spricht von einer 1,4 --- Verknüpfung. Es entsteht ein
        ungesättigtes Polymer $ \rightarrow$ Weitere Vernutzung möglich zum
        Elastomer oder Duroplast 
        %
        \begin{center}
            06.10.2020
        \end{center}
        %
        % Monomer und Polymer
        \begin{align*}
            % Monomer
            \schemestart
                \chemname{
                    \chemfig{
                        H
                        -[7]C
                            ( -[5]H)
                        =C
                            ( -[6]C
                            	( -[8]H)
                            	( -[6]H)
                            	( -[4]H)
                            )
                        -[0]C
                        	( =[1]\charge{90=\|, 0=\|}{O})
                        	( -[-1]\charge{45=\|, -135=\|}{O}
                                ( -[0]{CH3})
                        	)
                    }
                }
                {Monomer}
            \schemestop
        \end{align*}
        \begin{align*}
            % Polymer
            \schemestart
                \chemname{
                    \chemfig{
                        \ldots
                        -[0]C
                        	( -[2]H)
                        	( -[-2]H)
                        -C
                            ( -[2]{CH3})
                            (-[6]C
                            	( =[7]\charge{360=\|, 270=\|}{O})
                            	( -[5]\charge{315=\|, 135=\|}{O}
                            		 ( -[6]H)
                            	)
                            )
                        -[0]C
                        	( -[2]H)
                        	( -[-2]H)
                        -C
                            ( -[2]{CH3})
                            (-[6]C
                            	( =[7]\charge{360=\|, 270=\|}{O})
                            	( -[5]\charge{315=\|, 135=\|}{O}
                            		 ( -[6]H)
                            	)
                            )
                        -[0]C
                        	( -[2]H)
                        	( -[-2]H)
                        -C
                            ( -[2]{CH3})
                            (-[6]C
                            	( =[7]\charge{360=\|, 270=\|}{O})
                            	( -[5]\charge{315=\|, 135=\|}{O}
                            		 ( -[6]H)
                            	)
                            )
                        -\Charge{180=\, \ldots}{\vphantom{C}}
                    }
                }
                {Polymer}
            \schemestop
        \end{align*}
        Das ganze ist ein Thermoplast, weil es keine Verzweigung hat.\\
        %
        z.B. mit Styrol (Buna):
        \begin{align*}
            \schemestart
                \chemfig{
                    \ldots
                    -{CH2}
                    -{CH}
                        ( -[0]CH
                                ( -[2,1.5]C
                                	( -[4]H)
                                    ( -[0]**6(------))
                                    -[2]{CH2}
                                    -[2]\vdots
                                )
                            -{CH2}
                            -{CH2}
                            -{CH}
                                ( -[6]\vdots)
                            -{CH}
                                ( -[2,1.5]C
                                	( -[4]H)
                                    ( -[0]**6(------))
                                    -[2]{CH2}
                                    -[2]\vdots
                                )
                            -{CH2}
                            -\Charge{180=\, \ldots}{\vphantom{C}}
                        )
                    % connection
                    -[6]C
                    	( -[8]H)
                    	( -[4]H)
                    -[6]C
                    	( -[8]H)
                        ( -[4]**6(------))
                    -[6]C
                    	( -[8]H)
                    	( -[4]H)
                    -[6]C
                    	( -[8]H)
                        ( -[4]**6(------))
                    % row below
                        -[6,1.5]{CH}
                            -[4]\Charge{180=\, \ldots}{\vphantom{C}}
                        -{CH}
                        -{CH}
                            ( 
                                -[6,1.5]C
                                	( -[8]H)
                                    ( -[4]**6(------))
                                -[6]C
                                	( -[8]H)
                                	( -[4]H)
                                -[6]\vdots
                            )
                        -[0]C
                        	( -[2]H)
                        	( -[-2]H)
                        -[0]C
                        	( -[2]H)
                        	( -[-2]H)
                        -[0,2]{CH}
                            (-\Charge{0=\, \ldots}{\vphantom{C}})
                        -[2]C
                        	( -[4]H)
                            ( -[0]**6(------))
                        -[2]C
                        	( -[4]H)
                        	( -[0]H)
                        -[2]\vdots
                }
            \schemestop
        \end{align*}
        Die Verknüpfungen könnten beliebig lang sein und dadurch ist dieser
        Kunststoff elastisch.
        Je nach vernetzungsgrad bildet sich ein Elastromer oder ein Duroplast.
        \\
        Naturkautschuk: \\
        \begin{align*}
            \schemestart
                \chemname{
                    \chemfig{
                        {H2C}
                        =C
                            ( -[2]{CH3})
                        -{CH}
                        ={CH3}
                    }
                }
                {Polymer von Isopren}
            \schemestop
        \end{align*}
        Durch Vulkanisieren (Vernetzung durch Schwefelketten) ensteht Gummi.
        %
    \item Legosteine bestehen aus ABS (Acrylnitril --- Butadienstyrol) \\
        \begin{align*}
            \schemestart
                \chemname{
                    \chemfig{
                        % first
                        \ldots
                        -[@{l1}0]C
                        	( -[2]H)
                        	( -[-2]H)
                        -[0]C
                        	( -[2]H)
                        	( -[-2]N)
                        -[@{r1}0]
                        % second
                        -[@{l2}0]C
                        	( -[2]H)
                        	( -[-2]H)
                        -C
                            (-[2]H)
                        =C
                            (-[2]H)
                        -[0]C
                        	( -[2]H)
                        	( -[-2]H)
                        -[@{r2}0]
                        % third
                        -[@{l3}]
                        -C
                        	( -[2]H)
                            ( -[-2]**6(------))
                        -[0]C
                        	( -[2]H)
                        	( -[-2]H)
                        -[@{r3}]
                        \Charge{0=\, \ldots}{\vphantom{C}}
                    }
                    \polymerdelim[delimiters={[]}, height=45pt, depth=45pt, indice=n]{l1}{r1}
                    \polymerdelim[delimiters={[]}, height=45pt, depth=45pt, indice=m]{l2}{r2}
                    \polymerdelim[delimiters={[]}, height=45pt, depth=120pt, indice=l]{l3}{r3}
                }
                {Acrylnitril}
            \schemestop
        \end{align*}
        \begin{align*}
            \schemestart
                \chemname{
                    \chemfig{
                        H
                        -[7]C
                            (-[5]H)
                        =C
                            ( -[1]H)
                        -[7]C
                        ~\charge{0=\|}{N}
                    }
                }
                {Butadienstyrol}
            \schemestop
        \end{align*}
        Polymere, die aus verschiedenen Monomeren aufgebaut sind, nennt man
        Copolymere. Sie ermöglichen vielfältige Beeinflussung der Kunststoffe.
\end{enumerate}
%
%
% END
\end{document}
