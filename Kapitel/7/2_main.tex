\documentclass[../../main.tex]{subfiles}
\begin{document}
\subsection{Kunststofftypen}
%------------------
\subsubsection{Thermoplaste}
%------------------
% Eigenschaften
\underline{Eigenschaften:}
\begin{itemize}
    \item Werden beim \textit{erwärmen} \textbf{leicht} oder \textbf{schmelzen}.
    \item Lösen sich teilweise in Aceton oder quellen (aufquellen).
\end{itemize}
%------------------
Vorteile:
\begin{itemize}
    \item Gute Verarbeitungsmöglichkeiten:\\
        Schmelzen, dann pressen, spritzen, gießen (und extruhieren: Form auspressen(?))
    \item Gute Wiederverwertbarkeit: \\
        Einschmelzen der sortenreinen Kunststoffe.
\end{itemize}
%------------------
\underline{Skizze:}
\begin{figure}[ht]
    \centering
    \incfig{skizze_thermoplaste}
    \caption{Skizze von Thermoplaste}
    \label{fig:skizze_thermoplaste}
\end{figure}
\\
\underline{Erklärung:} \\
Sie bestehen aus linearen oder wenig verzweigten Makromolekülen und beim
erwärmen werden die Zwischenmolekularenkrüfte teilweise überwunden. \\
Die Ketten können aneinander vorbei gleiten. \\
Manche Lösungsmittel können sich zwischen den Ketten schieben $\rightarrow$
Kunststoff quillt auf oder löst sich auf.
%------------------
\subsubsection{Duroplaste, 21.09.2020}
Eigenschaften:
\begin{itemize}
    \item Zersetzen sich beim erwärmen, ohne zu schmelzen.
    \item unlöslich in Lösungsmitteln.
    \item Formbeständiger und widerstandsfähiger Kunststoff, \textbf{aber}:
        \begin{itemize}
            \item Schwer recyclebar
            \item schwer zu verarbeiten: Werkstücke müssen in der Form
                synthetisiert werden, anschließend nur mechanische Bearbeitung
                (Bohren, Sägen, Schleifen, Steckdosenabdeckung, etc.)
        \end{itemize}
\end{itemize}
\underline{Skizze:}
\begin{figure}[ht]
    \centering
    \incfig{skizze_duroplaste}
    \caption{Dreidimensionales Netz}
    \label{fig:skizze_duroplaste}
\end{figure}
%
\\
\underline{Erklärung:}
\begin{itemize}
    \item Duroplaste bestehen aus stark verzweigten Ketten, beim starkem
        erhitzen werden Atombindungen aufgebrochen $\rightarrow$ Der Stoff
        zersetzt sich. 
    \item  Manche Lösungsmittel schieben sich in das Netz, sodass manche
        Duroplaste aufquellen können.
\end{itemize}
%------------------
\subsubsection{Elastomere}
\underline{Eigenschaften:}
\begin{itemize}
    \item Biegbar/Elastisch und ist reversible (springt zurück in seine ursprüngliche Form)
    \item Beim erhitzen zersetzen ohne zu schmelzen.
\end{itemize}
%--
\newpage
\underline{Skizze:}
\begin{figure}[ht]
    \centering
    \incfig{skizze_elastomere}
    \caption{Skizze Elastomere}
    \label{fig:skizze_elastomere}
\end{figure}
%
\\
\underline{Erklärung:}
\\
Elastomere bestehen aus weitmaschtigen Makromolekülen. (Rest ist gleich wie Duroplaste)
%------------------
\newpage
\subsection{Herstellung von Kunststoffen}
\subsubsection{Polymerisation}
\underline{Versuch:} Herstellung von Polysterol\\
Skizze:
\begin{figure}[h]
    \centering
    \incfig{skizze-polymerisation}
    \caption{Skizze Polymerisation}
    \label{fig:skizze-polymerisation}
\end{figure}
\\
\underline{Beobachtung:}
\begin{itemize}
    \item Sidet beim erhitzen (auch wenn die Flamme weggenommen wird)
    \item Viskosität nimmt zu
    \item Schäumt beim siden
    \item aufsteigende Dämpfe, Kondensierun im Steigrohr
\end{itemize}
\underline{Definition:} Polymerisation \\
Verknüpfen kleiner Molekülen mit Doppelbindung zu einem Makromoleküle unter
Verlust der Doppelbindung.\\
%

\underline{Gesamtreaktion:}
\begin{align*}
    \schemestart
        \chemfig{
            **6(
                -
                -
                -
                -
                    (-[2]C
                        (-[3]H)
                        (=[0]C
                            (-[1]H)
                            (-[7]H)
                        )
                    )
                -
                -
            )
        }
        \arrow{->}
        \chemfig{
            -[@{BL}]
            -[0]C
                ( -[6]**6(------))
                ( -[2]H)
            -[0]C
           		( -[2]H)
           		( -[-2]H)
                -[@{BR}]
        }
        \polymerdelim[delimiters={[]}, height=45pt, depth=120pt,
        indice=n]{BL}{BR}
    \schemestop
\end{align*}
\textbf{\underline{Reaktionsmechanismus:}}
\\
\underline{1. Bildung von Radikalen:}
\begin{align*}
    \schemestart
        \chemname{
            \chemfig{[:90]
                **6(
                    -
                        (-[0]C
                        (=[2]\charge{45=\|,135=\|}{O})
                            (-[0]\charge{90=\|,270=\|}{O}
                                (-[0]\charge{90=\|,270=\|}{O}
                                    (-[0]C
                                        (=[2]\charge{45=\|,135=\|}{O})
                                        (-**6(------))
                                    )
                                )
                            )
                        )
                    -
                    -
                    -
                    -
                    -
                )
            }
        }
        {Dibenzoylperoxid}
    \schemestop
    \\
    \schemestart[][west]
        \arrow{->} 2 \hspace{1mm}
    \schemestop
    \schemestart[][west]
        \chemfig{[:90]
            **6(
                -
                    (-[0]C
                        (=[2]\charge{45=\|, 135=\|}{O})
                        (-[0]\charge{0=\., 90=\|, 270=\|}{O})
                    )
                -
                -
                -
                -
                -
            )
        }
    \schemestop
\end{align*}
Es spaltet sich auf, weil die Peroxidgruppe sehr instabil ist.
%
\\
\underline{2. Startreaktion}
\setchemfig{scheme debug=false}
\begin{align*}  
    \schemestart[][west]
    % first molecule
    \chemfig{[:30]
            **6(
                -
                -
                    (-[0]C
                        (=[2]\charge{45=\|, 135=\|}{O})
                        (-[0]\charge{0=\., 90=\|, 270=\|}{O})
                    )
                -
                -
                -
                -
            )
        }
        \arrow{0}[,0]
        \+{0.5cm, 0.5cm}
        \arrow{0}[,0]
        % second molecule
        \chemfig{
            H
            -[7]@{dest1}C
                (-[6]**6(------))
            =[@{double} 0]@{dest2}C
                ( -[1]H)
                ( -[7]H)
        }
    \schemestop
    % electron arrow
    \chemmove {
        \draw[shorten <= 5pt, shorten >= 5pt] (double) .. controls +(270:10mm) and +(180:10mm) .. (dest1);
        \draw[shorten <= 5pt, shorten >= 5pt] (double) .. controls +(270:10mm) and +(0:10mm) .. (dest2);
    }
    \\
    \schemestart
        \arrow{->}
    \schemestop
    % product
    \underbrace{
        \schemestart
            \chemfig{[:90]
                **6(
                    -
                        (-[0]C
                            (-[0]\charge{90=\|, 270=\|}{O}
                                (-[0]C
                                    (-[0]C)
                                    (-[2]H)
                                    (-[6]**6(------))
                                )
                            )
                            (=[2]\charge{45=\|, 135=\|}{O})
                        )
                    -
                    -
                    -
                    -
                    -
                )
            }
        \schemestop
    }_{R}
\end{align*}  
%
\underline{3. Kettenreaktion/Kettenwachstum:}
\begin{align*}
    \schemestart
        % educt
        R \+{0.5cm, 0.5cm}
        \arrow{0}[,0]
        % educt
        \chemfig{
            H 
            -[7]C
                ( -[6]**6(------))
            =[0]C
                ( -[1]H)
                ( -[7]H)
        }
        \arrow
        % product
        \chemfig{
            R
            -[0]C
                ( -[2]H)
                ( -[6]**6(------))
            -[0]\charge{0=\.}{C}
                ( -[2]H)
                ( -[6]H)
        }
    \schemestop
\end{align*}
%
\underline{4. Kettenabbruch:} \\
Verschiedene Möglickeiten, z.B. Rekombination:
\begin{align*}
    \schemestart
        % educt 1
        \chemfig{
            R_{1}
            -C
                ( -[2]H)
                ( -[6]**6(------))
            -\charge{0=\.}{C}
                ( -[2]H)
                ( -[6]H)
        }
        % ----
        \arrow{0}[,0]
        \+{1cm, 1cm}
        \arrow{0}[,0]
        % ----
        % educt 2
        \chemfig{
            \charge{180=\.}{C}
                ( -[2]H)
                ( -[6]H)
            -C
                ( -[2]H)
                ( -[6]**6(------))
            -R_{2}
        }
        % ----
        \arrow
        % ----
        \chemfig{
            R_{1}
            -C
                ( -[2]H)
                ( -[6]**6(------))
            -C
                ( -[2]H)
                ( -[6]H)
            -C
                ( -[2]H)
                ( -[6]H)
            -C
                ( -[2]H)
                ( -[6]**6(------))
            -R_{2}
        }
    \schemestop
\end{align*}
%
Dibenzoylperoxid ist hier Starter, beziehungsweise Radikalbildner und die Zugabe
von vielen Startern führt zu kürzeren Kettenlängen, da viele Ketten gestartet
werden. (Die Kette von der Gesamtreaktion)
%

\underline{Bemerkung / Beispiele zu Polymerisation}
\begin{enumerate}[label=\alph*)]
    %
    % a)
    \item Bekannte Polymerisation \\
        \noindent
        \begin{tabularx}{\linewidth}{|0l|0c|0c|>{\raggedright\arraybackslash\hspace{0pt}}X|}
            Name & Monomer & Polymermolekül & Einsatzbeispiel\\
            \hline
            \makecell[cl]{Polyethen\\ (PE)} 
                    & 
                    \adjustbox{valign=c}{\chemfig{
                        H
                        -[1]C 
                            (-[3]H)
                        =C 
                            ( -[1]H) 
                        -[7]H
                    }} &
                    \schemestart
                            \chemfig{
                                -[@{upleft,0.5},1]C
                                    ( -[2]H)
                                    ( -[-2]H)
                                -C
                                    ( -[2]H)
                                    ( -[-2]H)
                                -[@{upright,0.5},1]
                            }
                            \polymerdelim[delimiters ={[]}, height = 45pt, depth = 40pt, indice = n]{upleft}{upright}
                    \schemestop
                    & Plastiktüten
            \\
            \makecell[cl]{Polypropen\\ (PP)} 
                 & \adjustbox{valign=c}{\chemfig{
                     H
                     -C
                        ( -[2]H)
                        ( -[6]H)
                    -C
                        ( -[2]H)
                    =C
                        ( -[1]H)
                        ( -[7]H)
                 }} & 
                 \schemestart
                        \chemfig{
                            -[@{upleft,0.5},1]C
                                ( -[2]H)
                                ( -[-2]H)
                            -C
                                ( -[2]{CH_3})
                           -[@{right,0.5},1]
                        }
                        \polymerdelim[delimiters ={[]}, height = 45pt, depth = 40pt, indice = n]{upleft}{upright}
                 \schemestop
                 & Flaschendeckel, Brotdosen
            \\
            \makecell[cl]{Polyvinyl-\\chlorid\\ (PVC)}
                & 
                \adjustbox{valign=c}{\chemfig{
                    H
                    -[1]C
                        ( -[3]H)
                    =C
                        ( -[1]Cl)
                    -[7]H
                }} & 
                \schemestart
                        \chemfig{
                            -[@{upleft,0.5},1]C
                                ( -[2]H)
                                ( -[-2]H)
                            -C
                                ( -[6]H)
                                ( -[2]{Cl})
                            -[@{right,0.5},1]
                        }
                        \polymerdelim[delimiters ={[]}, height = 45pt, depth = 40pt, indice = n]{upleft}{upright}
                \schemestop
                & Rohrleitungen, Vinylböden, Schallplatten
            \\ 
            \makecell[cl]{Polytetra-\\fluorethen\\ (PTFE)} 
                & 
                \adjustbox{valign=c}{\chemfig{
                    F
                    -[1]C
                        ( -[3]F)
                    =C
                        ( -[1]F)
                    -[7]F
                }}&
                \schemestart
                        \chemfig{
                            -[@{upleft,0.5},1]C
                                ( -[2]F)
                                ( -[6]F)
                            -C
                                ( -[2]F)
                                ( -[6]F)
                            -[@{right,0.5},1]
                        }
                        \polymerdelim[delimiters ={[]}, height = 45pt, depth = 40pt, indice = n]{upleft}{upright}
                \schemestop
                & Pfannenbeschichtung (Teflon),
                Funktionskleidung (Goretex) 
        \end{tabularx}
    %
    % b)
    \item \underline{Amorph Teilkristallin} \\
        \begin{itemize}
            \item Amorphe Kunststoffe: Glasartig, transparent
            \item Teilkristalline Kunststoffe: Mechanisch Stabiler, nicht klar
                durchsichtig (milchig), wärmebeständig
        \end{itemize}
        % Amorphe Zeichnung
        \begin{figure}[ht]
            \centering
            \incfig{amorph_tteilkristallin_kristallin}
            \caption{amorph-teilkristallin-kristallin-Eigenschaften-Pfeile}
            \label{fig:amorph tteilkristallin kristallin}
        \end{figure}
    %
    % c)
    \item \underline{Weichmacher} \\
        Kleine Moleküle die sich zwischen die Ketten einlagern können $
        \rightarrow$ Mehr Abstand zwischen den Ketten $ \rightarrow$ Geringere
        zwischenmolekulare Kräfte zwischen den Ketten $ \rightarrow$
        Bessere Verschiebbarkeit der Ketten gegeneinander $ \rightarrow$
        Kunststoff wird weicher

        \underline{Problem:}\\
        \begin{itemize}
            \item Weichmachermoleküle können wieder leicht aus den Ketten
                rausgehen: \\
                %
                Weichmachermoleküle können schädlich sein für Mensch
                und Umwelt
            \item Weichmacher wird spröder, weil der Weichmacher raus ist
        \end{itemize}
    %
    % d)
    \item \underline{Monomere mit konjugierten Doppelbindungen} \\
        Bespiel:
        \begin{align*}
            \schemestart
                \chemname{
                    \chemfig{
                        CH3
                        -[1]{CH}
                            ( -[1]H)
                        =[7]C
                            ( -[5]H)
                            -[1]{CH3}
                    }
                }
                {1,3 --- Butdien}
            \schemestop
        \end{align*}
        %
        \begin{align*}
            \schemestart
                n
                \arrow{0}[,0]
                \chemfig{
                    H
                    -[7]@{l1}C
                        ( -[5]H)
                    =[@{main1}0]@{r1}C
                        ( -[1]H)
                    -[7]@{l2}C
                        ( -[5]H)
                    =[@{main2}0]@{r2}C
                        ( -[1]H)
                    -[7]H
                }
                \arrow{->}
                \chemfig{
                    -[@{BL}0]C
                		( -[2]H)
                		( -[-2]H)
                    -[0]C
                        ( -[2]H)
                    =[0]C
                        ( -[2]H)
                    -[0]C
                   		( -[2]H)
                   		( -[-2]H)
                    -[@{BR}0]
                }
                \polymerdelim[delimiters={[]}, height=45pt, depth=40pt,
                indice=n]{BL}{BR}
            \schemestop
            \chemmove {
                \draw[shorten <= 5pt, shorten >= 5pt] (main1) .. controls
                +(90:1cm) and +(180:1cm) .. (l1);
                \draw[shorten <= 5pt, shorten >= 5pt] (main1) .. controls
                +(90:1cm) and +(0:1cm) .. (r1);
                \draw[shorten <= 5pt, shorten >= 5pt] (main2) .. controls
                +(90:1cm) and +(180:1cm) .. (l2);
                \draw[shorten <= 5pt, shorten >= 5pt] (main2) .. controls
                +(90:1cm) and +(0:1cm) .. (r2);
            }
        \end{align*}
        % But 1,3 ---- dien
        Man spricht von einer 1,4 --- Verknüpfung. Es entsteht ein
        ungesättigtes Polymer $ \rightarrow$ Weitere Vernutzung möglich zum
        Elastomer oder Duroplast 
        %
        \begin{center}
            06.10.2020
        \end{center}
        %
        % Monomer und Polymer
        \begin{align*}
            % Monomer
            \schemestart
                \chemname{
                    \chemfig{
                        H
                        -[7]C
                            ( -[5]H)
                        =C
                            ( -[6]C
                            	( -[8]H)
                            	( -[6]H)
                            	( -[4]H)
                            )
                        -[0]C
                        	( =[1]\charge{90=\|, 0=\|}{O})
                        	( -[-1]\charge{45=\|, -135=\|}{O}
                                ( -[0]{CH3})
                        	)
                    }
                }
                {Monomer}
            \schemestop
        \end{align*}
        \begin{align*}
            % Polymer
            \schemestart
                \chemname{
                    \chemfig{
                        \ldots
                        -[0]C
                        	( -[2]H)
                        	( -[-2]H)
                        -C
                            ( -[2]{CH3})
                            (-[6]C
                            	( =[7]\charge{405=\|, 315=\|}{O})
                            	( -[5]\charge{360=\|, 180=\|}{O}
                                    ( -[6]{CH3})
                            	)
                            )
                        -[0]C
                        	( -[2]H)
                        	( -[-2]H)
                        -C
                            ( -[2]{CH3})
                            (-[6]C
                            	( =[7]\charge{405=\|, 315=\|}{O})
                            	( -[5]\charge{360=\|, 180=\|}{O}
                                    ( -[6]{CH3})
                            	)
                            )
                        -[0]C
                        	( -[2]H)
                        	( -[-2]H)
                        -C
                            ( -[2]{CH3})
                            (-[6]C
                            	( =[7]\charge{405=\|, 315=\|}{O})
                            	( -[5]\charge{360=\|, 180=\|}{O}
                                    ( -[6]{CH3})
                            	)
                            )
                    }
                }
                {Polymer}
            \schemestop
        \end{align*}
        %
        z.B. mit Styrol (Buna):
        \begin{align*}
            \schemestart
                \chemfig{
                    H
                }
            \schemestop
        \end{align*}
        Je nach vernetzungsgrad bildet sich ein Elastromer oder ein Duroplast.
        \\
        Naturkautschuk: \\
        Polymer von Isopren
        \begin{align*}
            \schemestart
                \chemfig{
                    {H2C}
                    =C
                        ( -[2]{CH3})
                    -{CH}
                    ={CH3}
                }
            \schemestop
        \end{align*}
        Durch Vulkanisieren (Vernetzung durch Schwefelketten) ensteht Gummi.
        %
    \item Legosteine bestehen aus ABS (Acrylnitril - Butadienstyrol) \\
        \begin{align*}
            \schemestart
                \chemname{
                    \chemfig{
                        H
                        -[7]C
                            (-[5]H)
                        =C
                            ( -[1]H)
                        -[7]C
                        ~\charge{0=\|}{N}
                    }
                }
                {Acrylnitril}
            \schemestop
        \end{align*}
        Polymere, die aus verschiedenen Monomeren aufgebaut sind, nennt man
        Copolymere. Sie ermöglichen vielfältige Beeinflussung der Kunststoffe.
\end{enumerate}
%
%
% END
\end{document}
