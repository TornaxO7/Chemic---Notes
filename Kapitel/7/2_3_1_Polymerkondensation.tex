\documentclass[../../main.tex]{subfiles}
\begin{document}
\subsection{7.3.2 Polykondensation}
Kondensationsreaktion:
\begin{center}
    Verknüpfung zweier Moleküle durch Abspaltung eines weiteren Moleküls (z.B.
    Wasser)
\end{center}
%
Bekannte Kondensationsreaktionen:
\begin{enumerate}[label=\alph*)]
    \item Esterbildung (Säure + Alkohol)
    \item Peptidbildung (aus Aminosäuren)
\end{enumerate}
%
Struktureformeln zu a) und b):
%
\begin{enumerate}[label=\alph*)]
    \item \underline{Polyester}

        Möglichkeit 1: Hydroxycarbonsäure\\
        z.B.:
        \begin{align*}
            \schemestart
                n
                \hspace{0.5cm}
                \chemfig{
                    H
                    -\charge{90=\|, -90=\|}{O}
                    -C
                    	(-[2]H)
                    	(-[-2]H)
                    -[0]C
                    	(=[1]\charge{90=\|, 0=\|}{O})
                    	(-[-1]\charge{60=\|, -120=\|}{O}
                    		 (-[0]H)
                    	)
                }
            \schemestop
            \\
            \\
            \schemestart
            \arrow{->}
            \ldots
            \chemfig{
                -[@l]\charge{90=\|, -90=\|}{O}
                -[0]C
                	(-[2]H)
                	(-[-2]H)
                -C
                    (=[2]\charge{135=\|, 45=\|}{O})
                -[@r]\Charge{180=$\ldots$}{\vphantom{C}}
            }
            \polymerdelim[delimiters={[]}, height=45pt, depth=40pt, indice=n]{l}{r}
            \+
            n
            \ce{H2O}
            \schemestop
        \end{align*}
        Möglichkeit 2: Dicarbonsäure + Dialkohol
        \begin{align*}
            \schemestart
                \chemfig{
                    -[4]C
                    	( =[5]\charge{270=\|, 180=\|}{O})
                    	( -[3]\charge{240=\|, 60=\|}{O}
                    		 ( -[4]H)
                    	)
                    -[0]C
                    	( -[2]H)
                    	( -[-2]H)
                    -[0]C
                    	( =[1]\charge{90=\|, 0=\|}{O})
                    	( -[-1]\charge{60=\|, -120=\|}{O}
                    		 ( -[0]H)
                    	)
                }
                \+
                \chemfig{
                    H
                    -\charge{90=\|, -90=\|}{O}
                    -[0]C
                    	( -[2]H)
                    	( -[-2]H)
                    -[0]C
                    	( -[2]H)
                    	( -[-2]H)
                    -\charge{90=\|, -90=\|}{O}
                    -H
                }
            \schemestop
            \\
            \\
            \schemestart
                \arrow{->[Kondensation]}
                \qquad
                \chemfig{
                    -[@l]C
                        (=[2]\charge{135=\|, 45=\|}{O})
                    -C
                    	( -[2]H)
                    	( -[-2]H)
                    -C
                    	( =[1]\charge{90=\|, 0=\|}{O})
                    -[-1]\charge{45=\|, -135=\|}{O}
                    -C
                    	( -[2]H)
                    	( -[-2]H)
                    -C
                    	( -[2]H)
                    	( -[-2]H)
                    -\charge{90=\|, -90=\|}{O}
                    -[@r]
                }
                \polymerdelim[delimiters={[]}, height=45pt, depth=120pt, indice=n]{l}{r} 
                \arrow{0}[,0]
                \+
                2n
                \ce{H2O}
            \schemestop
        \end{align*}
\end{enumerate}
%
%
% END
\end{document}
