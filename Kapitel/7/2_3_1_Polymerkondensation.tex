\documentclass[../../main.tex]{subfiles}
\begin{document}
\subsection{7.3.2 Polykondensation}
Kondensationsreaktion:
\begin{center}
    Verknüpfung zweier Moleküle durch Abspaltung eines weiteren Moleküls (z.B.
    Wasser)
\end{center}
%
Bekannte Kondensationsreaktionen:
\begin{enumerate}[label=\alph*)]
    \item Esterbildung (Säure + Alkohol)
    \item Peptidbildung (aus Aminosäuren)
\end{enumerate}
%
Struktureformeln zu a) und b):
%
\begin{enumerate}[label=\alph*)]
    \item \underline{Polyester}

        Möglichkeit 1: Hydroxycarbonsäure\\
        z.B.:
        \begin{align*}
            \schemestart
                n
                \hspace{0.5cm}
                \chemfig{
                    H
                    -\charge{90=\|, -90=\|}{O}
                    -C
                    	(-[2]H)
                    	(-[-2]H)
                    -[0]C
                    	(=[1]\charge{90=\|, 0=\|}{O})
                    	(-[-1]\charge{60=\|, -120=\|}{O}
                    		 (-[0]H)
                    	)
                }
            \schemestop
            \\
            \\
            \schemestart
            \arrow{->}
            \ldots
            \chemfig{
                -[@l]\charge{90=\|, -90=\|}{O}
                -[0]C
                	(-[2]H)
                	(-[-2]H)
                -C
                    (=[2]\charge{135=\|, 45=\|}{O})
                -[@r]\Charge{180=$\ldots$}{\vphantom{C}}
            }
            \polymerdelim[delimiters={[]}, height=45pt, depth=40pt, indice=n]{l}{r}
            \+
            n
            \ce{H2O}
            \schemestop
        \end{align*}
        Möglichkeit 2: Dicarbonsäure + Dialkohol
        \begin{align*}
            \schemestart
                \chemfig{
                    -[4]C
                    	( =[5]\charge{270=\|, 180=\|}{O})
                    	( -[3]\charge{240=\|, 60=\|}{O}
                    		 ( -[4]H)
                    	)
                    -[0]C
                    	( -[2]H)
                    	( -[-2]H)
                    -[0]C
                    	( =[1]\charge{90=\|, 0=\|}{O})
                    	( -[-1]\charge{60=\|, -120=\|}{O}
                    		 ( -[0]H)
                    	)
                }
                \+
                \chemfig{
                    H
                    -\charge{90=\|, -90=\|}{O}
                    -[0]C
                    	( -[2]H)
                    	( -[-2]H)
                    -[0]C
                    	( -[2]H)
                    	( -[-2]H)
                    -\charge{90=\|, -90=\|}{O}
                    -H
                }
            \schemestop
            \\
            \\
            \schemestart
                \arrow{->[Kondensation]}
                \qquad
                \chemfig{
                    -[@l]C
                        (=[2]\charge{135=\|, 45=\|}{O})
                    -C
                    	( -[2]H)
                    	( -[-2]H)
                    -C
                    	( =[1]\charge{90=\|, 0=\|}{O})
                    -[-1]\charge{45=\|, -135=\|}{O}
                    -C
                    	( -[2]H)
                    	( -[-2]H)
                    -C
                    	( -[2]H)
                    	( -[-2]H)
                    -\charge{90=\|, -90=\|}{O}
                    -[@r]
                }
                \polymerdelim[delimiters={[]}, height=45pt, depth=45pt, indice=n]{l}{r} 
                \arrow{0}[,0]
                \+
                2n
                \ce{H2O}
            \schemestop
        \end{align*}
    \item 
        \begin{align*}
            \schemestart
                n
                \chemfig{
                    H
                    -[-1]\charge{45=\|, -135=\|}{O}
                    -[-1]C
                        (=[1]\charge{135=\|, -45=\|}{O})
                    -[6]C
                    	( -[0]\charge{90=\|, -90=\|}{O}
                            (-H)
                        )
                    	( -[4]H)
                    -[6]C
                    	(-[8]H)
                    	(-[6]H)
                    	(-[4]H)
                }
                \arrow{->}
                \chemfig{
                    -[@l]C
                        (=[2]\charge{135=\|, 45=\|}{O})
                    -[0]C
                    	( -[2]H)
                    	( -[-2]{\ce{CH3}})
                    -\charge{90=\|, -90=\|}{O}
                    -[@r]
                }
                \polymerdelim[delimiters={[]}, height=45pt, depth=45pt, indice=n]{l}{r} 
                \+
                n \ce{H2O}
            \schemestop
        \end{align*}

    \item Polyamid (PA)

        \underline{Versuch:} Herstellung von Nylon

        Lösung A:
        \begin{itemize}
            \item $2{,}2g$ 1,6 -- Diaminohexan und
            \item $1g$ \ce{NaOH} in $50ml$ Wasser
        \end{itemize}

        Lösung B:
        \begin{itemize}
            \item $1{,}5 ml$ Dekansäuredichlorid
            \item in $50ml$ Heptan
        \end{itemize}

        Lösung A wird mit Lösung B überschichtet.

        Mit einer Pinzette lässt sich Grenzschicht als Faden herausziehen.

        \underline{(1. Möglichkeit) Erklärung:}

        Aus Diamin und Dicarbonsäure bzw. Dicarbonsäurechlorid entsteht ein
        Polyamid:

        \begin{align*}
            \schemestart
                \arrow{0}[0,0]
                \chemfig{
                    \charge{135=\|, -135=\|, -45=\|}{\ce{Cl}}
                    -[1]C
                        (=[3]\charge{180=\|, 90=\|}{O})
                    -[@l]{\ce{CH2}}
                    -[@r]C
                        (=[1]\charge{90=\|, 0=\|}{O})
                    -[-1]\charge{45=\|, -45=\|, -135=\|}{\ce{Cl}}
                }
                \polymerdelim[delimiters={()}, indice=8]{l}{r} 
                \arrow{0}[0,0]
                \+
                n
                \chemfig{
                    H
                    -\charge{90=\|}{N}
                        (-[-2]H)
                    -{\ce{CH2}}
                    -[@l]{\ce{CH2}}
                    -[@r]{\ce{CH2}}
                    -\charge{90=\|}{N}
                        (-[-2]H)
                    -H
                }
                \polymerdelim[delimiters={()}, height=15pt, depth=15pt, indice=4]{l}{r} 
            \schemestop
            \\
            \\
            \schemestart
                \arrow{->}
                \chemfig{
                    -[@l]C
                        (=[2]\charge{135=\|, 45=\|}{O})
                    -[@{l2}]{\ce{CH2}}
                    -[@{r2}]C
                        (=[2]\charge{135=\|, 45=\|}{O})
                    -\charge{90=\|}{N}
                        (-[-2]H)
                    -[@{l3}]{\ce{CH2}}
                    -[@{r3}]\charge{90=\|}{N}
                        (-[-2]H)
                    -[@r]
                }
                \polymerdelim[delimiters={[]}, height=45pt, depth=45pt, indice=n]{l}{r} 
                \polymerdelim[delimiters={()}, indice=8]{l2}{r2} 
                \polymerdelim[delimiters={()}, indice=6]{l3}{r3} 
                \+
                2n \ce{HCL_{(aq)}}
            \schemestop
        \end{align*}
        Hier ist eine Aminogruppe, weil dort eine Peptidbindung ist!

        \underline{2. Möglichkeit der Polyamidsynthese: Aminosäuren}

        Technisch meist: Aminoruppe und Carboxylgruppe endständig.

        Variante: Vorgelagerte intramolekulare Kondenstationsreakion.

        Beispiel: Perlon
        \begin{align*}
            \schemestart
                n
                \chemfig{
                    H
                    -\charge{90=\|}{N}
                        (-[-2]H)
                    -C
                        (=[6]\charge{315=\|, 225=\|}{O})
                    -[@l]{\ce{CH2}}
                    -[@r]C
                    	(=[1]\charge{90=\|, 0=\|}{O})
                    	(-[-1]\charge{60=\|, -120=\|}{O}
                    		 ( -[0]H)
                    	)
                }
                \polymerdelim[delimiters={()}, height=15pt, depth=15pt, indice=5]{l}{r} 
                \arrow{<=>}
            \schemestop
            \\
            \\
            \schemestart
                n
                \arrow{0}[0,0]
                \hspace{0.5cm}
                \chemfig{
                    [:180]
                    \ce{CH2}*7(
                    -\ce{CH2}
                    -\ce{CH2}
                    -\charge{0=\|}{N}
                        (-[4]H)
                    -C
                        (=[1]\charge{90=\|, 0=\|}{O})
                    -\ce{CH2}
                    -\ce{CH2}
                    -\ce{CH2}
                    )
                }
                \arrow{0}[0,0]
                \+
                n
                \hspace{0.25cm}
                \ce{H2O_{(aq)}}
                \arrow{->}
                \chemname{
                    \chemfig{
                        -[@l]\charge{90=\|}{N}
                            (-[6]H)
                        -[@{l2}]\ce{CH2}
                        -[@{r2}]C
                            (=[2]\charge{135=\|, 45=\|}{O})
                        -[@r]
                    }
                    \polymerdelim[delimiters={[]}, height=45pt, depth=45pt, indice=n]{l}{r} 
                }
                {Perlon}
            \schemestop
        \end{align*}
\end{enumerate}

%SUBFILE

\subsubsection{Beeinflussung des Kunststofftyps der Polykondensation}
\begin{itemize}
    \item \textbf{Thermoplast}:

        Bei Difunktionellen Monomeren entstehen lineare Ketten also eine
        Thermoplast.
        \begin{itemize}
            \item Variante 1: Eine Hydroxycarbonsäure
            \item Variante 2: Dicarbonsäure und Dialkohol

                \underline{Nachteil (zur Variante 2):}

                Genaues abgestimmes Verhältnis erforderlich! Denn ansonsten
                würden nur kurze Ketten entstehen.
        \end{itemize}

    \item \textbf{Elastomer} und \textbf{Duroplast}:

        Durch Beimischung von trinofunktionellen Monomere ergibt sich eine
        Vernetzung und je nach Menge der trinofunktionellen Monomere ein
        \textit{Elastomer} oder ein \textit{Duroplast}.
\end{itemize}

Verwendet man zur Polykondensation ungesättigte Verbindungen (Verbindungen
mit einer Doppelbindung) wie:
\begin{align*}
    \chemname{
        \chemfig{
            H
            -\charge{90=\|, -90=\|}{O}
            -C
                (-[2]H)
            =C
                (-[-2]H)
            -C
            	(=[1]\charge{90=\|, 0=\|}{O})
            	(-[-1]\charge{60=\|, -120=\|}{O}
            		 ( -[0]H)
            	)
        }
    }
    {3 -- Hydroxypropensäure}
\end{align*}
so kann man den entstehenden Thermoplastischen Polyester anschließend durch
Polymerisation zum Duroplast vernetzen. Sowas nennt man auch
\textit{Polyesterharz}.
%
%
% END
\end{document}
