\documentclass[../main.tex]{subfiles}
\begin{document}
\subsection{Elastomere}
\underline{Eigenschaften:}
\begin{itemize}
    \item Biegbar/Elastisch und ist reversible (springt zurück in seine ursprüngliche Form)
    \item Beim erhitzen zersetzen ohne zu schmelzen.
\end{itemize}
%--
\newpage
\underline{Skizze:}
\begin{figure}[ht]
    \centering
    \incfig{skizze_elastomere}
    \caption{Skizze Elastomere}
    \label{fig:skizze_elastomere}
\end{figure}
%
\\
\underline{Erklärung:}
\\
Elastomere bestehen aus weitmaschtigen Makromolekülen. (Rest ist gleich wie Duroplaste)
%------------------
\begin{merke}{Eselsbrücke}
    \textbf{Elasto}mere sind \textbf{elasti}sch.
\end{merke}
%
%
% END
\end{document}
