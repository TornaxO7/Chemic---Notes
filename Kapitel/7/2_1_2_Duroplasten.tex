\documentclass[../../main.tex]{subfiles}
\begin{document}
\subsection{Duroplaste}
Eigenschaften:
\begin{itemize}
    \item Zersetzen sich beim erwärmen, ohne zu schmelzen.
    \item unlöslich in Lösungsmitteln.
    \item Formbeständiger und widerstandsfähiger Kunststoff, \textbf{aber}:
        \begin{itemize}
            \item Schwer recyclebar
            \item schwer zu verarbeiten: Werkstücke müssen in der Form
                synthetisiert werden, anschließend nur mechanische Bearbeitung
                (Bohren, Sägen, Schleifen, Steckdosenabdeckung, etc.)
        \end{itemize}
\end{itemize}
\underline{Skizze:}
\begin{figure}[ht]
    \centering
    \incfig{skizze_duroplaste}
    \caption{Dreidimensionales Netz}
    \label{fig:skizze_duroplaste}
\end{figure}
%
\\
\underline{Erklärung:}
\begin{itemize}
    \item Duroplaste bestehen aus stark verzweigten Ketten, beim starkem
        erhitzen werden Atombindungen aufgebrochen $\rightarrow$ Der Stoff
        zersetzt sich. 
    \item  Manche Lösungsmittel schieben sich in das Netz, sodass manche
        Duroplaste aufquellen können.
\end{itemize}
%
%
% END
\end{document}
