\documentclass[../../main.tex]{subfiles}
\begin{document}
\subsection{Thermoplaste}
%------------------
% Eigenschaften
\underline{Eigenschaften:}
\begin{itemize}
    \item Werden beim \textit{erwärmen} \textbf{leicht} oder \textbf{schmelzen}.
    \item Lösen sich teilweise in Aceton oder quellen (aufquellen).
\end{itemize}
%------------------
Vorteile:
\begin{itemize}
    \item Gute Verarbeitungsmöglichkeiten:\\
        Schmelzen, dann pressen, spritzen, gießen (und extruhieren: Form auspressen(?))
    \item Gute Wiederverwertbarkeit: \\
        Einschmelzen der sortenreinen Kunststoffe.
\end{itemize}
%------------------
\underline{Skizze:}
\begin{figure}[ht]
    \centering
    \incfig{skizze_thermoplaste}
    \caption{Skizze von Thermoplaste}
    \label{fig:skizze_thermoplaste}
\end{figure}
\\
\underline{Erklärung:} \\
Sie bestehen aus linearen oder wenig verzweigten Makromolekülen und beim
erwärmen werden die Zwischenmolekularenkrüfte teilweise überwunden. \\
Die Ketten können aneinander vorbei gleiten. \\
Manche Lösungsmittel können sich zwischen den Ketten schieben $\rightarrow$
Kunststoff quillt auf oder löst sich auf. \\
\begin{merke}{Eselsbrücke}
    \textbf{Thermo}plaste verformen sich bei hoher
    \textbf{Temperatur}.
\end{merke}
%
%
% END
\end{document}
