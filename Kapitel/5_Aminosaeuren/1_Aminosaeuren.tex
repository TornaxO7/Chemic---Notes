\documentclass[../../main.tex]{subfiles}
\begin{document}
Allgemeine Struktur:
\begin{align*}
    \schemestart
        \chemname{
            \chemfig{
                C
                	( =[3]\charge{180=\|, 90=\|}{O})
                	( -[1]\charge{150=\|, -30=\|}{O}
                		 ( -[2]H)
                	)
                -[6]C
                    ( -[4]\charge{180=\|}{N}
                        ( -[2]H)
                        ( -[6]H)
                    )
                    ( -[0]H )
                -[6]R
            }
        }
        {L --- $\alpha$ --- Aminosäure}
    \schemestop
\end{align*}
%------------------
\subsection{Namensbedeutungen}
Das $\alpha$ steht für die \textit{Carboxylgruppe am benachbartem C---Atom}.

Aminosäuren liegen als \textbf{Zwitter} vor.
\begin{itemize}
    \item Durch \textbf{Carboxylgruppe}: Kann \textbf{Sauer (Protonendonator)}
        reagieren.
    \item Durch \textbf{Aminogruppe}: Kann \textbf{Basisch (Protonenakzeptor)}
        reagieren.
\end{itemize}
Es bildet durch die beiden Gruppen eine \textbf{intramolekulare
    Protonenwanderung}.

%------------------
\begin{tabularx}{\linewidth}{
    |>{\centering\arraybackslash}X
    |>{\centering\arraybackslash}X
    |>{\centering\arraybackslash}X
    |}
    \hline
    Kation
    &
    Zwitterion
    &
    Anion
    \\ % ----------
    \hline
    \schemestart
        \chemfig{
            COOH
            -[6]C
                ( -[4]{\ce{H3N^{+}}} )
                ( -[0]H )
            -[6]H
        }
    \schemestop
    &
    \schemestart
        \chemfig{
            COO^{-}
            -[6]C
                ( -[4]{\ce{H3N^{+}}} )
                ( -[0]H )
            -[6]H
        }
    \schemestop
    &
    \schemestart
        \chemfig{
            COO^{-}
            -[6]C
                ( -[4]\ce{H2N} )
                ( -[0]H )
            -[6]H
        }
    \schemestop
    \\
    \hline
\end{tabularx}
%------------------

Den pH---Wert, an dem die Aminosäuren hauptsächlich als Zwitterion vorliegen
nennt man \textbf{isoelektrischen Punk (IEP)}.
%
\subsection{Peptidbindung}
Bei einer Peptidbindung spalten sich ein Sauerstoff von der Carboxylgruppe und
zwei Wasserstoff Atome von der Aminogruppe ab, sodass Wasser entsteht.
Anschließend verbunden sie sich:
\begin{align*}
    \schemestart
        \chemname{
            \chemfig{
                H
                -\charge{45:5pt=$\oplus$}{N}
                    ( -[2]H)
                    ( -[6]H)
                -[0]C
                	( -[2]H)
                	( -[-2]H)
                -[0]C
                	( =[1]\charge{90=\|, 0=\|}{O})
                -[7]\charge{90=\|, 0=\|, -90=\|, 45:5pt=$\ominus$}{O}
            }
        }
        {Glycin}
        \+
        \chemname{
            \chemfig{
                H
                -N
                    ( -[2]H)
                    ( -[6]H)
                -C
                	( -[2]H)
                    ( -[-2]{CH3})
                -C
                	( =[1]\charge{90=\|, 0=\|}{O})
                -[7]\charge{90=\|, 0=\|, -90=\|, 45:5pt=$\ominus$}{O}
            }
        }
        {Alanin}
    \schemestop
\end{align*}
%
% END
\end{document}
